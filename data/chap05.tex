% !TeX root = ../thuthesis-example.tex

\chapter{Ensemble method}

\section{Introduction}

In the literature, very few methods offer sequence-based and structure-based methods together
and usually focus on one aspect or the other. However, methods such as REF1 and REF2 show
that an approach composed of both sequential and structural features can be extremely
befenecial for specific tasks.

In this chapter, we present the first ensemble method that uses both sequential and 
structural information. It outperforms 3 out of the 4 current-state-of-the art models and
outperforms the best model, ProSmith, on the cold protein and cold substrate subset, which
is of significant importance for drug design. 

\section{Methods}

For this ensemble method, we are building upon our existing sequence-based and structure-based models and we
experimented several methods. To do so, we used our trained model to predict the validation set. We then used
the predictions on the validation set as training samples for different methods such as trees and linear
regression, where we observed the best results: a MSE of 0.636 and a $r^2$ of 0.540. More specifically, we
obtained:

\begin{table}[ht]
  \centering
  \begin{tabular}{lcccccc}
  \hline
   & \multicolumn{3}{c}{\textbf{Hot}} & \multicolumn{3}{c}{\textbf{Cold}} \\
   & Samples & MSE & R\(^2\) & Samples & MSE & R\(^2\) \\ \hline
  \textbf{Hot seq}  & 1192 & 0.563 & 0.549 & 64 & 0.742 & 0.502 \\
  \textbf{Cold seq} & 985 & 0.682 & 0.553 & 98 & 1.091 & 0.124 \\ \hline
  \end{tabular}
  \caption{Ensemble method results on the test set}
  \label{tab:summary_performance}
\end{table}

And for the new test set, we obtained a MSE of 0.995 and a $r^2$ of 0.435:

\begin{table}[ht]
  \centering
  \begin{tabular}{lcccccc}
  \hline
   & \multicolumn{3}{c}{\textbf{Hot}} & \multicolumn{3}{c}{\textbf{Cold}} \\
   & Samples & MSE & R\(^2\) & Samples & MSE & R\(^2\) \\ \hline
  \textbf{Hot seq}  & 40 & 0.952 & 0.362 & 30 & 0.731 & -0.080 \\
  \textbf{Cold seq} & 1117 & 0.908 & 0.464 & 102 & 2.032 & 0.011 \\ \hline
  \end{tabular}
  \caption{Ensemble method results on the new test set}
  \label{tab:summary_performance}
\end{table}
