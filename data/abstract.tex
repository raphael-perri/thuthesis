% !TeX root = ../thuthesis-example.tex

% 中英文摘要和关键字

\begin{abstract}
  论文的摘要是对论文研究内容和成果的高度概括。
  摘要应对论文所研究的问题及其研究目的进行描述,对研究方法和过程进行简单介绍,对研究成果和所得结论进行概括。
  摘要应具有独立性和自明性,其内容应包含与论文全文同等量的主要信息。
  使读者即使不阅读全文,通过摘要就能了解论文的总体内容和主要成果。

  论文摘要的书写应力求精确、简明。
  切忌写成对论文书写内容进行提要的形式,尤其要避免“第 1 章……;第 2 章……;……”这种或类似的陈述方式。

  关键词是为了文献标引工作、用以表示全文主要内容信息的单词或术语。
  关键词不超过 5 个,每个关键词中间用分号分隔。

  % 关键词用“英文逗号”分隔,输出时会自动处理为正确的分隔符
  \thusetup{
    keywords = {关键词 1, 关键词 2, 关键词 3, 关键词 4, 关键词 5},
  }
\end{abstract}

\begin{abstract*}
  Enzymes are specific proteins that accelerate the rates of reactions. 
  Almost all metabolic reactions in cells need to be catalyzed in order to occur at 
  rates fast enough to substain life. Hence, understanding the mechanisms by which enzymes
  catalyze reactions is of the utmost importance not only to understand life but also to
  create new enzymes with enhance properties. A possible method is to the study the 
  Michaelis constant $K_m$, the substrate concentration at which the reaction velocity
  is 50\% of the maximum velocity. This can help measure the affinity of an enzyme with
  its substrate, the molecule it catalyze the reaction of. To do so, recent methods have
  shown promissing results using sequence-based model such as large language models for
  proteins. However, structure-based methods have not been extensively research. In this
  work, we evaluate both methods and build a combined model using both the sequence and 
  the structure information to predict the Michaelis constant. Our model outperforms
  the state-of-the-art methods for completely unseen proteins and substrates. This model
  can then be used as a scoring model to evaluate newly design pr

  % Use comma as separator when inputting
  \thusetup{
    keywords* = {keyword 1, keyword 2, keyword 3, keyword 4, keyword 5},
  }
\end{abstract*}
