% !TeX root = ../thuthesis-example.tex

\chapter{Discussion}
\label{chap:6}

\section{Introduction}

In this chapter, we will not only examinate our methods and results but will also provide a deep dive into
ProSmith, the current state-of-the-art model for Michaelis constant prediction. We will also take a deeper look
into the dataset used for this task and finally offer some perspective on the evaluation metrics and the use
of Michaelis constant predictive models.

\section{ProSmith Analysis}

While ProSmith offers the results in terms of MSE and correlation coefficient $r^2$, we also wanted to 
explore its ability to model the interactions between the protein and the substrate, as the Michaelis constant
is a reflection of it. As we noticed, its performance on completely unseen proteins (cold proteins) and
completely unseen substrates (cold substrates) is very low and we wanted to know if the general results were
due to a real understanding of the protein and substrate interaction or simply an overfit on the data, which
is very common with small datasets in biology-related fields.

To do so, we conducted 2 main experiements. The first one, the single amino acid mutation effect analysis,
aims at determining the ability of the model to deal with different but very similar proteins. 
The second experiment's goal is to see where the Michaelis constant distribution lies between different 
substrates.

\section{Dataset}

All the four models that were presented here use the same dataset curated by \textcolor{red}{xxx}. As we 
built a new test dataset using data from 2022 to 2024, we had the opportunity to look into details into how
the dataset was built and have noted some possible improvements.

As a reminder, the dataset is coming from the BRENDA database, which is a comprehensive compilation of information 
on enzymes and their biochemical properties, including the Michaelis constant.

One problem of this data preprocessing is the way the protein sequences are obtained. For 6,882 out of the
11,675 total entries (59\%), the UniProtID is available and hence it is very simple to retrieve the protein
sequence. However, for the 40\% remaining, the UniProtID is not accessible. The authors of \textcolor{red}{paper}
hence decided to use the EC Number and organism in order to retrieve a sequence.

EC Numbers provide a systematic way to classify enzymes based on the chemical reactions they catalyze. 
By using EC numbers in combination with organism names, we can identify enzymes of interest even when 
specific identifiers like UniProt IDs are not available. Using the organism on top of this can help to narrow
down the possible sequences. However, the same EC number can correspond to multiple isozymes with varying 
sequences within or across species. Therefore, using the EC number and organism name might retrieve multiple 
sequences, complicating the identification of the specific enzyme of interest and making its link with the
Michaelis constant, which can drastically vary across different enzymes of the same EC Number and organism, can
lead to misleading results.

Hence, this method offer sequences for 40\% of the dataset which we have no information about how close they
are to reality and can dramatically impact the results. It is therefore difficult to classify the results of
all these models for real applications as it may not reflect experimental results, and hence, not be as 
useful as it could be.

To deal with this problem, we also curated a completely new dataset for the Michaelis constant from the
SABIO-RK database, which, similarly to BRENDA, is a repository that contains information about biochemical 
reactions and their kinetic properties. "Sabio-RK" stands for "System for the Analysis of Biochemical 
Pathways - Reaction Kinetics." It is designed to store, organize, and provide access to experimental 
kinetic data extracted from literature. The database is particularly useful for researchers in the fields 
of biochemistry, enzymology, and systems biology, as it offers a wealth of data that can be used for 
modeling and simulation of metabolic pathways, enzyme kinetics, and for the design of biological experiments.
While BRENDA is more general, the SABIO-RK database focuses on the kinetics of biochemical reaction, like 
the Michaelis constant and therefore provide more specific and reliable data for our work.

In the end, we curated a dataset of over 27,000 protein-substrate pairs, containing not only wildtype like the
previously existing dataset but also mutants that can be of great significance for analyzing kinetic models
ability to capture protein and substrate interactions. This dataset can be used for training our model but we
decided to focus on different specificities that we will discuss in the next section.

\section{Model Evaluation and Perspectives}

In the Michaelis constant prediction literature, the most common metrics are the MSE and the correlation
coefficient, as shown by the four state-of-the-art models presented in the work. 

While these metrics are a very good indication of the model overall performance and ability to predict a 
specific value based on the protein-substrate pairs, it may not be the most adequate to model the interaction
between the enzyme and its substrate. Indeed, as we saw with ProSmith, while it performs best on the evaluation
metrics, it does not seem to capture the interaction between the enzyme and its substrate and fail to provide
Michaelis constant values that are realistic.

Furthermore, while the Michaelis constant can be very interesting to understand a specific protein-substrate
pair, predicting it for many unknown pairs might lack real-world significance. However, it can become
extremely valuable if we decide to use as a scoring function. Indeed, as the Michaelis constant measures the
affinity of an enzyme with its substrate, we can use it to score mutants for specific wildtype and accelerate
the drug discovery process. 

To produce more efficient enzymes in enzyme engineering and screening, understanding and manipulating 
enzyme-substrate interactions is crucial. By predicting Km values for a vast 
array of mutant enzymes against their respective substrates, we can rank these mutants 
according to their efficiency and selectivity. This approach enables the prioritization of 
candidates that demonstrate improved performance over wild types, streamlining the enzyme engineering process.

In drug discovery, identifying molecules that can modulate enzyme activity is a core strategy. 
By using Km as a scoring function, scientists can assess how different drug candidates affect 
enzyme-substrate affinity. This method allows for the rapid screening of large libraries of compounds 
to identify those with the potential to either inhibit or enhance enzyme activity, thus acting as 
effective drugs or lead compounds.

Finally, the ability to predict Michaelis constant for enzyme-substrate pairs where experimental data 
is lacking opens new options in targeting diseases at the molecular level. For instance, if a mutant enzyme 
is associated with a particular disease, understanding its substrate affinity could reveal novel 
therapeutic strategies. These predictions could guide the design of substrate-mimicking drugs or 
competitive inhibitors, offering targeted interventions with potentially fewer side effects.

In summary, while predicting the Michaelis constant for various protein-substrate pairs presents challenges, 
its application as a scoring function in drug discovery and enzyme engineering is very promising.